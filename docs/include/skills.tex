\section{Skills}

Bogboa is a \emph{skills} based RPG engine.  You can think of a skill as both
a category for player abilities and rating of effectiveness when performing
those abilities in that category.  For example, the ability to mend the wounds
of others might fall within the skill \emph{healing}.  A character with a 
higher healing skill would naturally heal better than one with a lower rating
while a third character with no healing skill could not learn the ability in
the first place.

That last point is important.  In Bogboa, you don't limit abilities to
specific classes.  You specifty a minimum skill level required to learn an 
ability and any character who meets or exceeds that skill can obtain it.
\footnote{They may have to fetch a scroll from the bottom of a dungeon or 
pay a trainer hundreds of gold, however.}

This means that if you gave one class a tiny conjuration skill, say .1, 
they could learn conjuration spells at 1/10th the ability (and level) of a
class with a conjuration skill of 1.0.  Giving a class a skill modifier of
zero insures they will never advance it.

Skills are \emph{capped} per character level.  The formula for this is:

\[ Level \times 10 \times Class Skill Rating \]
      
For your own sanity (and balance reasons), I recommend that you use
a rating of 1.0 for class-defining skills.  Give the best swordfighting
class a `sword skill' of 1.0 and adjust other swordsmen relative to that.
You might be tempted to create something like a cavalier class with
exceptional swordfighting and grant them a 1.4 skill rating.  To illustrate
the problem this creates, let's imagine a level 10 cavalier player fighting
an equal level troll.  The troll has a moderate 0.8 clubbing skill.  Now we
have the player doing the dps of a level 14 fighter against a mob doing the
dps of a level 8.  Sure, you \emph{might} be able to balance this by making the
cavalier a paper tiger but, again, I suggest starting out conservatively.
    
\subsection{Advancing Skills}

Players advance skills by \emph{using} them.  Freshly created characters begin
with skills at zero.  Each time they perform an ability, a skill-up roll is
performed by the server.  Each successful roll raises the skill by one point
until the current level cap is met.  Rolls are weighted where the upper points
are harder to achieve.

Obviously, you must provide players access to abilities in skill categories
intended for their class. For example, if the first `psychic' skill requires
20 points to learn then no player can get it.  You can either provide trainer
NPCs that offer basic, rating zero skills or grant base abilities to each class 
(or both).  

\subsection{Non Player Characters}

NPCs use the same skill modifiers for their class as players but are
automatically granted max skills for their assigned level.  It is important to
remember that skill \emph{tweaking} effects monsters as well. 
    

 

    
